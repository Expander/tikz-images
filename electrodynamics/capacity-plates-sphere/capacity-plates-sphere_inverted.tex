\documentclass[crop,tikz]{standalone}
\usetikzlibrary{backgrounds}
\colorlet{blue}{cyan}
\tikzset{
  inverted/.style = {
    color=white,
    background rectangle/.style={fill},
    show background rectangle
  }
}
\usepackage{pgfplots}
\pgfplotsset{compat=1.13}

\pgfplotsset{
  inverted/.style = {
    every axis legend/.append style={
      draw=white,
      fill=black,
      text=white
    }
  },
}

\begin{document}
\begin{tikzpicture}[inverted,inverted]
  \pgfmathsetmacro{\Radius}{1}; % radius of sphere
  \pgfmathsetmacro{\EO}{1}; % field strength of plate capacitor
  \pgfmathsetmacro{\er}{4}; % relative permittivity
  \pgfmathsetmacro{\A}{-3*\EO/(\er + 2)};
  \pgfmathsetmacro{\B}{(\er - 1)/(\er + 2)*\Radius^3*\EO};
  \pgfmathsetmacro{\scalingfactor}{3};
  \pgfmathsetmacro{\minx}{-\Radius*\scalingfactor};
  \pgfmathsetmacro{\maxx}{\Radius*\scalingfactor};
  \pgfmathsetmacro{\miny}{-\Radius*\scalingfactor};
  \pgfmathsetmacro{\maxy}{\Radius*\scalingfactor};
  \def\myvalues{-3.,-2.8,-2.6,-2.4,-2.2,-2.,-1.8,-1.6,-1.4,-1.2,-1.,-0.8,-0.6,-0.4,-0.2,0.,0.2,0.4,0.6,0.8,1.,1.2,1.4,1.6,1.8,2.,2.2,2.4,2.6,2.8,3.}
  %
  \begin{axis}[inverted,
    axis equal image,
    axis lines = none,
    xmin={\minx}, xmax={\maxx},
    ymin={\miny}, ymax={\maxy},
    view={0}{90},
    samples=50,
    samples y=50,
    clip=false,
    declare function={
      % potential inside the sphere
      % Vin[x_,y_,z_] := A z
      Vin(\x,\y,\z) = \A*\z;
      % potential outside the sphere
      % Vout[x_,y_,z_] := -E0 z + B LegendreP[1,z/Sqrt[x^2+y^2+z^2]]/(x^2+y^2+z^2)
      Vout(\x,\y,\z) = -\EO*\z + \B*\z/sqrt(\x^2 + \y^2 + \z^2)^3;
      % field lines inside the sphere
      % Ein[x_,y_,z_] := -A x
      Ein(\x,\y,\z) = -\A*\x;
      % field lines outside the sphere
      % Eout[x_,y_,z_] := E0 x + B*x/Sqrt[x^2 + y^2 + z^2]^3
      Eout(\x,\y,\z) = \EO*\x + \B*\x/sqrt(\x^2 + \y^2 + \z^2)^3;
    }
    ]
    % field lines
    % inside sphere
    \addplot3[
      domain={-\Radius}:{\Radius},
      domain y={-\Radius}:{\Radius},
      contour gnuplot={levels=\myvalues,draw color=red,labels=false},
      very thin,
      x filter/.expression={sqrt(x^2+y^2) < \Radius ? x : nan},
      y filter/.expression={sqrt(x^2+y^2) < \Radius ? y : nan}] { Ein(x,0,y) };
    % outside sphere
    \addplot3[
      domain={\minx}:{\maxx},
      domain y={0}:{\maxy},
      contour gnuplot={levels=\myvalues,draw color=red,labels=false},
      very thin,
      x filter/.expression={sqrt(x^2+y^2) > \Radius ? x : nan},
      y filter/.expression={sqrt(x^2+y^2) > \Radius ? y : nan}] { Eout(x,0,y) };
    \addplot3[
      domain={\minx}:{\maxx},
      domain y={\miny}:{0},
      contour gnuplot={levels=\myvalues,draw color=red,labels=false},
      very thin,
      x filter/.expression={sqrt(x^2+y^2) > \Radius ? x : nan},
      y filter/.expression={sqrt(x^2+y^2) > \Radius ? y : nan}] { Eout(x,0,y) };
    % lines of constant potential
    % inside sphere
    \addplot3[
      domain={-\Radius}:{\Radius},
      domain y={-\Radius}:{\Radius},
      contour gnuplot={levels=\myvalues,draw color=blue,labels=false},
      very thin,
      x filter/.expression={sqrt(x^2+y^2) < \Radius ? x : nan},
      y filter/.expression={sqrt(x^2+y^2) < \Radius ? y : nan}] { Vin(x,0,y) };
    % outside sphere
    \addplot3[
      domain=0:{\maxx},
      domain y={\miny}:{\maxy},
      contour gnuplot={levels=\myvalues,draw color=blue,labels=false},
      very thin,
      x filter/.expression={sqrt(x^2+y^2) > \Radius ? x : nan},
      y filter/.expression={sqrt(x^2+y^2) > \Radius ? y : nan}] { Vout(x,0,y) };
    \addplot3[
      domain={\minx}:0,
      domain y={\miny}:{\maxy},
      contour gnuplot={levels=\myvalues,draw color=blue,labels=false},
      very thin,
      x filter/.expression={sqrt(x^2+y^2) > \Radius ? x : nan},
      y filter/.expression={sqrt(x^2+y^2) > \Radius ? y : nan}] { Vout(x,0,y) };
    % capacitors
    \draw[thick] ({\minx},{\miny}) -- ({\maxx},{\miny});
    \draw[thick] ({\minx},{\maxy}) -- ({\maxx},{\maxy});
    \draw[thick] (0,0) circle ({\Radius});
  \end{axis}
\end{tikzpicture}
\end{document}
