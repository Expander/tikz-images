\documentclass[crop,tikz]{standalone}
\usetikzlibrary{backgrounds}
\colorlet{blue}{cyan}
\tikzset{
  inverted/.style = {
    color=white,
    background rectangle/.style={fill},
    show background rectangle
  }
}
\usepackage{pgfplots}
\pgfplotsset{compat=1.18}

% Field lines and lines of constant potential for four quarter
% cylinders of radius 1 with center at (0,0). The top right and bottom
% left cylinders have constant potential V, while the top left and
% bottom right cylinders have constant potential -V.
%
% See also Jackson problem 2.14.
%
% We take the complex potential to be
% phi[z_, b_:1, V_:1] := 2V/Pi Log[(z^2 - b^2)/(z^2 + b^2)]
%
% With z = r Exp[I t] we obtain the real potential
% phirt[r_, t_, b_:1, V_:1] := 2V/Pi ArcTan[2 r^2 b^2 Sin[2t], r^4 - b^4]

\colorlet{green}{green}

\pgfplotsset{
  inverted/.style = {
    every axis legend/.append style={
      draw=white,
      fill=black,
      text=white
    }
  }
}

\begin{document}

\begin{tikzpicture}[inverted,inverted]
  \pgfmathsetmacro{\Radius}{1}; % radius of capacitor
  \pgfmathsetmacro{\numberofpotentiallines}{30};
  \pgfmathsetmacro{\remin}{-4};
  \pgfmathsetmacro{\remax}{-\remin};
  \pgfmathsetmacro{\immin}{-4};
  \pgfmathsetmacro{\immax}{-\immin};
  \pgfmathsetmacro{\cmin}{-30}; % minimum c
  \pgfmathsetmacro{\cmax}{30}; % maximum c
  \begin{axis}[inverted,
    axis equal image,
    xmin={\remin}, xmax={\remax},
    ymin={\immin}, ymax={\immax},
    xlabel={$\Re$},
    ylabel={$\Im$},
    view={0}{90},
    declare function = {
      % lines of constant potential
      f(\p,\c) = sqrt(\c*sin(2*\p) + sqrt((\c*sin(2*\p))^2 + 1)); % r(phi)
      fc(\n,\nmax) = \cmin + \n/\nmax*(\cmax - \cmin); % calculates c
      % field lines
      g(\x,\y) = (\x^4 + 2*\x^2*(-1 + \y^2) + (1 + \y^2)^2)/(\x^4 + (-1 + \y^2)^2 + 2*\x^2*(1 + \y^2));
    },
    ]
    % lines of constant potential
    \pgfplotsinvokeforeach{0,...,{\numberofpotentiallines}}{
      \addplot[blue,very thin,domain={0}:{360},samples=400] (
        {\Radius*f(x,fc(#1,\numberofpotentiallines))*cos(x)},
        {\Radius*f(x,fc(#1,\numberofpotentiallines))*sin(x)}
      );
    }
    % electric field lines
    \addplot3[
      very thin,
      samples=100,
      domain={\remin}:{\remax},
      domain y={\immin}:{\immax},
      contour gnuplot={
        levels={1./10, 1./5, 3./10, 2./5, 1./2, 3./5, 7./10, 4./5, 9./10, 10./9, 5./4, 10./7, 5./3, 2, 5./2, 10./3, 5, 10},
        draw color=red,
        labels=false
      }] { g(x,y) };
    % capacitor
    \draw[thick] (0:\Radius) arc (0:90:\Radius) node at (45:{1.25*\Radius}) {\scriptsize $V$};
    \draw[thick,green] (90:\Radius) arc (90:180:\Radius) node at (135:{1.25*\Radius}) {\scriptsize $-V$};
    \draw[thick] (180:\Radius) arc (180:270:\Radius) node at (225:{1.25*\Radius}) {\scriptsize $V$};
    \draw[thick,green] (270:\Radius) arc (270:360:\Radius) node at (315:{1.25*\Radius}) {\scriptsize $-V$};
  \end{axis}
\end{tikzpicture}
\end{document}
