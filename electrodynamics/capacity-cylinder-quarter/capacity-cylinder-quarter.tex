\documentclass[crop,tikz]{standalone}
\usepackage{pgfplots}
\pgfplotsset{compat=1.18}

% Lines of constant potential for four quarter cylinders of radius 1
% at position (x,y) = (0,0). The top right and bottom left cylinders
% have constant potential V > 0, while the top left and bottom right
% cylinders have constant potential -V.
%
% See also Jackson problem 2.14.
%
% We take the potential to be
% phirt[r_, t_, b_:1, V_:1] := 2V/Pi ArcTan[2 r^2 b^2 Sin[2t], r^4 - b^4]

\colorlet{green}{black!40!green}

\pgfplotsset{
  inverted/.style = {
    every axis legend/.append style={
      draw=white,
      fill=hardblack,
      text=white
    }
  }
}

\begin{document}

\begin{tikzpicture}
  \pgfmathsetmacro{\Radius}{1}; % radius of capacitor
  \pgfmathsetmacro{\numberoffieldlines}{30};
  \pgfmathsetmacro{\remin}{-4};
  \pgfmathsetmacro{\remax}{-\remin};
  \pgfmathsetmacro{\immin}{-4};
  \pgfmathsetmacro{\immax}{-\immin};
  \pgfmathsetmacro{\cmin}{-30}; % minimum c
  \pgfmathsetmacro{\cmax}{30}; % maximum c
  \begin{axis}[
    axis equal image,
    xmin={\remin}, xmax={\remax},
    ymin={\immin}, ymax={\immax},
    xlabel={$\Re$},
    ylabel={$\Im$},
    samples=400,
    declare function = {
      f(\p,\c) = sqrt(\c*sin(2*\p) + sqrt((\c*sin(2*\p))^2 + 1)); % r(phi)
      fc(\n,\nmax) = \cmin + \n/\nmax*(\cmax - \cmin); % calculates c
    },
    ]
    % lines of constant potential
    \pgfplotsinvokeforeach{0,...,{\numberoffieldlines}}{
      \addplot[blue,very thin,domain={0}:{360}] (
        {\Radius*f(x,fc(#1,\numberoffieldlines))*cos(x)},
        {\Radius*f(x,fc(#1,\numberoffieldlines))*sin(x)}
      );
    }
    % capacitor
    \draw[thick] (0:\Radius) arc (0:90:\Radius) node at (45:{1.25*\Radius}) {\scriptsize $V$};
    \draw[thick,green] (90:\Radius) arc (90:180:\Radius) node at (135:{1.25*\Radius}) {\scriptsize $-V$};
    \draw[thick] (180:\Radius) arc (180:270:\Radius) node at (225:{1.25*\Radius}) {\scriptsize $V$};
    \draw[thick,green] (270:\Radius) arc (270:360:\Radius) node at (315:{1.25*\Radius}) {\scriptsize $-V$};
  \end{axis}
\end{tikzpicture}
\end{document}
