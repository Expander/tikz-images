\documentclass[crop,tikz]{standalone}
\usepackage{pgfplots}
\pgfplotsset{compat=1.18}

% Flow around a circle with non-zero circulation.
%
% Derived by trasforming the uniform flow f(w) = v*w via the Joukowski
% mapping J(z) = z + 1/z and adding a vortex term -i*k*ln(z). The
% resulting complex potential is given by g(z) = v*(z + r^2/z) - i*k*ln(z).
% v corresponds to the velocity, r is the radius of the circle and k
% describes the circulation. The flow lines are given by
% Im(g(z)) = constant = c.

\pgfplotsset{
  inverted/.style = {
    every axis legend/.append style={
      draw=white,
      fill=hardblack,
      text=white
    }
  },
}

\begin{document}
\begin{tikzpicture}
  \pgfmathsetmacro{\remin}{-4};
  \pgfmathsetmacro{\remax}{4};
  \pgfmathsetmacro{\immin}{-4};
  \pgfmathsetmacro{\immax}{4};
  \pgfmathsetmacro{\velocity}{1};
  \pgfmathsetmacro{\radius}{1};
  \pgfmathsetmacro{\circlulation}{2}; % try 1.6, 2 or 2.5
  % generated with: min = -8; max = 3; Subdivide[min, max, (max - min)*5] // N
  \def\contourlevels{-8., -7.8, -7.6, -7.4, -7.2, -7., -6.8, -6.6, -6.4, -6.2, -6., -5.8, -5.6, -5.4, -5.2, -5., -4.8, -4.6, -4.4, -4.2, -4., -3.8, -3.6, -3.4, -3.2, -3., -2.8, -2.6, -2.4, -2.2, -2., -1.8, -1.6, -1.4, -1.2, -1., -0.8, -0.6, -0.4, -0.2, 0., 0.2, 0.4, 0.6, 0.8, 1., 1.2, 1.4, 1.6, 1.8, 2., 2.2, 2.4, 2.6, 2.8, 3.}
  \begin{axis}[
    axis equal image,
    xmin={\remin}, xmax={\remax},
    ymin={\immin}, ymax={\immax},
    xlabel={$\Re$},
    ylabel={$\Im$},
    view={0}{90},
    samples=100,
    samples y=100,
    ]
    % restrict flow lines to rectangular area w/o inner circle
    \begin{scope}[even odd rule]
      \clip ({\remin},{\immin}) rectangle ({\remax},{\immax}) (0,0) circle (\radius);
      % flow lines
      \addplot3[
        domain={\remin}:{\remax},
        domain y={\immin}:{\immax},
        contour gnuplot={levels=\contourlevels,draw color=red,labels=false},
        very thin] { \velocity*y*(1 - \radius^2/(x^2 + y^2)) - 0.5*\circlulation*ln(x^2 + y^2) };
    \end{scope}
    % circle
    \draw[ultra thick] (0,0) circle (\radius);
  \end{axis}
\end{tikzpicture}
\end{document}
