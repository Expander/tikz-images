\documentclass[crop,tikz]{standalone}
\usetikzlibrary{backgrounds}
\colorlet{blue}{cyan}
\tikzset{
  inverted/.style = {
    color=white,
    background rectangle/.style={fill},
    show background rectangle
  }
}

\usepackage{amsmath}
\usepackage{pgfplots}
\usepackage{pgfplotstable}
\tikzset{>=latex}
\pgfplotsset{compat=1.16}

% rainbow line
\pgfdeclareverticalshading{rainbow}{100bp}
{color(0bp)=(red); color(25bp)=(red); color(35bp)=(yellow);
color(45bp)=(green); color(55bp)=(cyan); color(65bp)=(blue);
color(75bp)=(violet); color(100bp)=(violet)}

\begin{document}
\begin{tikzpicture}[inverted,inverted]
  % data generated with: data/strahlungsspektrum.m
  \pgfplotstableread{data/T300K.dat}\tableThreeHoundred
  \pgfplotstableread{data/T500K.dat}\tableFiveHoundred
  \pgfplotstableread{data/T1000K.dat}\tableOneThousand
  \pgfplotstableread{data/T3000K.dat}\tableThreeThousand
  \pgfplotstableread{data/T5777K.dat}\tableFiveThousand
  \pgfplotstableread{data/T10000K.dat}\tableTenThousand
  \pgfplotstableread{data/wien.dat}\tableWien
  \begin{loglogaxis}[
    axis on top, % axis before the rectangle
    width=10cm,
    height=8cm,
    xmin=0.1, xmax=100,
    ymin=1e-2, ymax=1e10,
    xlabel={$\lambda/\mu\text{m}$},
    ylabel={$S_\lambda(\lambda)/(\text{W}\,\text{m}^{-2}\,\mu\text{m}^{-1})$}
    ]
    % light band
    \shade[shading=rainbow,shading angle=90,opacity=0.2] (axis cs:0.38,1e-2) rectangle (axis cs:0.78,1e10);
    % intensities
    \addplot[thick]     table [x index=0, y index=1] {\tableThreeHoundred};
    \addplot[thick]     table [x index=0, y index=1] {\tableFiveHoundred};
    \addplot[thick]     table [x index=0, y index=1] {\tableOneThousand};
    \addplot[thick]     table [x index=0, y index=1] {\tableThreeThousand};
    \addplot[thick,red] table [x index=0, y index=1] {\tableFiveThousand};
    \addplot[thick]     table [x index=0, y index=1] {\tableTenThousand};
    % temperatures
    \node[rotate=66,font=\small,anchor=west] at (2.8 ,2e-2) {$300\,\text{K}$};
    \node[rotate=70,font=\small,anchor=west] at (1.4 ,2e-2) {$500\,\text{K}$};
    \node[rotate=75,font=\small,anchor=west] at (0.6 ,2e-2) {$1000\,\text{K}$};
    \node[rotate=77,font=\small,anchor=west] at (0.16,2e-2) {$3000\,\text{K}$};
    \node[rotate=70,font=\small,red]         at (0.15,1e+5) {$5777\,\text{K}$};
    \node[rotate=40,font=\small]             at (0.18,5e+8) {$10000\,\text{K}$};
    % Wien's displacement law
    \addplot[blue,thick,dashed] table [x index=0, y index=1] {\tableWien};
    \draw[->,blue] (rel axis cs: 0.95,0.95) node[above,anchor=north east,align=center,fill=white] {Wien'sches\\ Verschiebungsgesetz} -- (2.5,2e5);
  \end{loglogaxis}
\end{tikzpicture}
\end{document}
